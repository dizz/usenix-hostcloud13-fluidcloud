\documentclass[letterpaper,twocolumn,10pt]{article}
\usepackage{usenix,epsfig,endnotes}

\begin{document}
\date{}

\title{\Large \bf FluidCloud: An Open Framework for Relocation of Cloud Services}

\author{
{\rm Andy Edmonds}\\
Z\"urcher Hochschule f\"ur Angewandte Wissenschaften
\and
{\rm Thijs Metsch}\\
Intel Ireland Limited
\and
{\rm Dana Petcu}\\
Institute e-Austria Timisoara
\and
{\rm Erik Elmroth}\\
Ume{\aa} University
\and
{\rm Jamie Marshall}\\
Prologue
\and
{\rm Plamen Ganchosov}\\
CloudSigma
}

\maketitle

% \thispagestyle{empty}

\subsection*{Abstract}

Cloud computing delivers new levels of being connected, instead of the once disconnected 
PC-type systems. The proposal in this paper extends that level of connectedness in the cloud such that 
cloud service instances, hosted by providers, can relocate between clouds. This is key in order to 
provide economical and regulatory benefits but more importantly liberation and positive market disruption.

While service providers want to lock in their customer's services, FluidCloud wants the liberation of those and thereby allow the service owner to freely choose the best matching provider at any time. In the cloud world of 
competing cloud standards and software solutions, each only partially complete, the central research 
question which this paper intends to answer:
\noindent
\textbf{How to intrinsically enable and fully automate relocation of service instances between clouds?}


